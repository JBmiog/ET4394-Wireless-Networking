\chapter{Implementation}

\section{overview} 
The simulation consist of multiple runs, each run corresponds to pre-determined amount of access points $k$ and stations $n$ as depicted in figure \ref{fig:NS3}. After each run, the average throughput is measured. 
 

\begin{figure}[h]
\centering
\includegraphics[width=0.7\linewidth]{figures/NS3}
\caption[Simulation setup]{}
\label{fig:NS3}
\end{figure}

\section{detailed description of the script} 
The simulator script is heavily based upon the wifi-wired-bridging.cc file that can be found in the examples/wireless folder of the NS3 installation. The nodes are arbitrarily instantiated using the corresponding containers and helpers, which, for the purpose of this simulation, are placed in a c++ std::vector variable for easy scalability. The AP's that form the backbone architecture of this simulation are suited with a CSMA NetDevice, and form a CSMA network. This is choosen as the CSMA implementation in NS3 are instantaneous, faster than light, and transmission do not get jammed \cite{CSMADevices}. Thus, during the simulation we can purely focus on the 802.11b, and can leave IEEE 802.3 inter AP communication issues out of scope. 

YansWifiHelper and YansChanneHelper are used in order to make the connection between AP and stations, as well as defining the properties of the channels.

Next, the Mobility helper is used to let the nodes move around in a pre-defined square, and the stations are connected to the AP's. Next IPv4 packets are generated by an application which is installed on each node, that will transmit these packets to the access point $k$. 

Unfortunatly, the Random Mobility Model does not 

The average throughput of all nodes is calculated using FlowMonitor.  

The simulation is run with 30 stations, having 30, 15, 10, 6, 5 and 2 Access Points respectively. The DataRate is set to a constant of 5000kb/s.    
