\chapter{Introduction}

\section{assignment description} 

This assignment is one of two assignments given in the course of Wireless Networks [4393] at the Technical University of Delft. The assignment consist of writing a simulation in C++, using the NS-3 Network simulator, in order to simulate IEEE802.11b properties with respect to changing network properties. 

\section{The NS-3 Network simulator} 
NS-3 is a discreet-event network simulator, which is free to download at \url{https://www.nsnam.org/}. It contains models for Wi-Fi and Wi-Max, among others. The software is licensed under the GNU GPLv2 license. Although simulation files are  originally written in c++, recent features also support python scripts. The simulator supports WiFi simulation up to layer 3.   

\section{802.11b specification} 
802.11b operates in the 2.4 GHz band and is capable of transferring data with rates up to 11Mbps. It uses the CSMA/CA technique to let nodes communicate. It is Complementary Code Keyed (CCK) modulated, which is based on Direct Sequence Spread Spectrum modulation (DSSS) \cite{802.11b}.