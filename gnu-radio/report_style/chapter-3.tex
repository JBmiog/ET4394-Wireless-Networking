\chapter{Transmitter side}

Two transceivers will be used during this project, both are set to transmitting mode only. They are named 'Buenos aires' and 'Pijnacker' respectivily. The transmitter settings are show in the table below.

\begin{tabular}{|c|c|c|c|c|c|c|}
	\hline   & Tx Freq & Deviation & Mode & Tx Data & baudrate & Tx power \\ 
	\hline Pijacker & 432.9mHz & +/- 20kHz & FSK & 'Pijnacker' & 1200 symb/sec & -8dBm\\ 
	\hline Buenos Aires & 433.62 & +/- 20kHz & FSK & 'Buenos Aires' & 1200 symb/sec & -8dBm\\ 
	\hline 
\end{tabular} 

The signal that will be transmitted will contain a preamble of 4 bytes (0xAA).  This is followed by the transmitter identifiers, which are the same for both modules. The identifier is "TMCS" (0x54, 0x4D, 0x43, 0x53), followed by a node address byte (0xXX). 